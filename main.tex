\documentclass[12pt]{article}

% Поддержка Unicode и шрифтов
\usepackage{fontspec}
\usepackage{polyglossia}

% Язык документа
\setmainlanguage{russian}

% Основной шрифт (с поддержкой кириллицы)
\setmainfont{DejaVu Serif}

% Моноширинный шрифт для кириллицы в \texttt{}
\newfontfamily{\cyrillicfonttt}{DejaVu Sans Mono}

% Метаданные
\title{Пример документа с кириллицей}
\author{Студент}
\date{\today}

\begin{document}

\maketitle

\section{Введение}

Этот документ демонстрирует работу \LaTeX{} с кириллическими символами.
Здесь используются:
\begin{itemize}
	\item Шрифт \texttt{DejaVu Serif},
	\item Движок \texttt{xelatex},
	\item Пакеты \texttt{fontspec} и \texttt{polyglossia}.
\end{itemize}
\section{Заключение}

Всё работает корректно: буквы «абвгдеёжзийклмнопрстуфхцчшщъыьэюя» отображаются без ошибок.

Поддержка переносов включена благодаря пакету \texttt{texlive-lang-cyrillic}.

\end{document}
